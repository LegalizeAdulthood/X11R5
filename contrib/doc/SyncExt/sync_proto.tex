% Copyright 1991 by Olivetti Research Limited, Cambridge, England and
% Digital Equipment Corporation, Maynard, Massachusetts.
%
%                        All Rights Reserved
%
% Permission to use, copy, modify, and distribute this software and its 
% documentation for any purpose and without fee is hereby granted, 
% provided that the above copyright notice appear in all copies and that
% both that copyright notice and this permission notice appear in 
% supporting documentation, and that the names of Digital or Olivetti
% not be used in advertising or publicity pertaining to distribution of the
% software without specific, written prior permission.  
%
% DIGITAL AND OLIVETTI DISCLAIM ALL WARRANTIES WITH REGARD TO THIS SOFTWARE,
% INCLUDING ALL IMPLIED WARRANTIES OF MERCHANTABILITY AND FITNESS, IN NO EVENT
% SHALL THEY BE LIABLE FOR ANY SPECIAL, INDIRECT OR CONSEQUENTIAL DAMAGES OR
% ANY DAMAGES WHATSOEVER RESULTING FROM LOSS OF USE, DATA OR PROFITS, WHETHER
% IN AN ACTION OF CONTRACT, NEGLIGENCE OR OTHER TORTIOUS ACTION, ARISING OUT
% OF OR IN CONNECTION WITH THE USE OR PERFORMANCE OF THIS SOFTWARE.

\documentstyle{article}

\setlength{\parindent}{0 pt}
\setlength{\parskip}{6pt}

\newcommand{\system}[1]{{\sc #1}}
\newcommand{\request}[1]{{\bf #1}}
\newcommand{\event}[1]{{\bf #1}}

\newcommand{\param}[1]{{\it #1}}
\newcommand{\error}[1]{{\bf #1}}
\newcommand{\enum}[1]{{\bf #1}}

\newcommand{\eventdef}[1]{\item \event{#1}}
\newcommand{\requestdef}[1]{\item \request{#1}}
\newcommand{\errordef}[1]{\item \error{#1}}
\newcommand{\systemdef}[1]{\item \system{#1}}

\newcommand{\defn}[1]{{\bf #1}}

\begin{document}

\begin{center}

{\large X SYNCHRONIZATION EXTENSION}\\[10pt]
{\large Protocol Version 2.0}\\[15pt]
{\it Tim Glauert}\\[0pt]
{\tt thg@cam-orl.co.uk}\\[0pt]
{\bf Olivetti Research/MultiWorks}\\[5pt]
{\it Dave Carver}\\[0pt]
{\tt dcc@athena.mit.edu}\\[0pt]
{\bf Digital Equipment Corporation,}\\[0pt]
{\bf MIT / Project Athena}\\[5pt]
{\it Jim Gettys}\\[0pt]
{\tt jg@crl.dec.com}\\[0pt]
{\bf Digital Equipment Corporation,}\\[0pt]
{\bf Cambridge Research Laboratory}\\[5pt]

\end {center}

\subsection*{Synchronization}

The core X protocol makes no guarantees about the relative order of execution
of requests for different clients. This means that any synchronization between
clients must be done at the client level in an operating-system dependent and
network dependent manner. Even if there was an accepted standard for such
synchronization, the use of a network introduces unpredictable delays between
the synchronization of the clients and the delivery of the resulting requests
to the X server.

The core X protocol also makes no guarantees about the time at which requests
are executed, which means that all clients with real-time constraints must
implement their timing on the host computer. Any such timings are subject to
error introduced by delays within the operating system and network, and are
inefficient due to the need for round-trip requests which keep the client and
server synchronized.

The synchronization extension provides primitives which allow synchronization
between clients to take place entirely within the X server. This removes any
error introduced by the network and makes it possible to synchronize clients
on different hosts running different operating systems. This is important for
multi-media applications where audio, video and graphics data streams are
being synchronized. The extension also provides internal timers within the X
server to which client requests can be synchronized. This allows simple
animation applications to be implemented without any round-trip requests and
to make best use of buffering within the client, network and server.

\subsection*{Description}

The mechanism used by this extension for synchronization within the X server
is to block the processing of requests from a client until a specific
synchronization condition occurs. When the condition occurs the client is
released and processing of requests continues. Multiple clients may block on
the same condition to give inter-client synchronization.  Alternatively, a
single client may block on a condition such as an animation frame-marker.

The extension adds a \defn{Counter} to the set of resources managed by the
server. A counter has a single integer value which may be increased but may
not be reduced. A client can block by sending an \request{Await} request which
waits for a counter to reach a specific value.  The client is released when
the counter is advanced to that value.  A counter may be advanced by an
explicit \request{AdvanceCounter} request, which allows inter-client
synchronization.

There is a set of counters, called \defn{System Counters}, which are advanced
by the server itself rather than by an external client. The extension provides
a system counter which advances with the server time as defined by the core
protocol, and it may also provide counters which advance with the real-world
time or which advance each time the CRT screen is refreshed.  Other extensions
may provide their own extension-specific system counters.

The extension provides an \defn{Alarm} mechanism which allows clients to
receive an event whenever a particular counter is advanced.

\subsection*{Types}

The following new types are used by the extension.

\begin{tabular}{l}
	CARD64: 64-bit unsigned integer\\
	COUNTER: XID\\
	SYSTEMCOUNTER:\\
\begin{tabular}{@{\hspace*{1cm}}l}
	[\param{name}:STRING8,\\
	 \param{counter}:COUNTER,\\
	 \param{resolution}:CARD64]
\end{tabular}\\
        VALUETYPE: \{\enum{Absolute},\enum{Relative}\}\\
	WAITCONDITION:\\
\begin{tabular}{@{\hspace*{1cm}}l}
	[\param{counter}:COUNTER,\\
	 \param{wait-type}:VALUETYPE,\\
	 \param{wait-value}:CARD64,\\
	 \param{event-threshold}:CARD64]
\end{tabular}\\
	ALARM: XID\\
	ALARMSTATE: \{\enum{Active},\enum{Exhausted},\enum{Destroyed}\}\\
\end{tabular}

COUNTER is the client-side handle on a server \defn{Counter}. The value of a
counter is represented as a CARD64.

The SYSTEMCOUNTER type provides the client with information about a
\defn{System Counter}. The \param{name} is the textual name of the counter.
The \param{counter} is the client-side handle which should be used in
\request{Await} requests. The \param{resolution} field is a indication
of how much a system counter advances when it is updated. The
extension will not reliably resolve two \request{Await} requests that
differ by less than \param{resolution}.

The following system counters are defined by this extension and
should be implemented where possible.

\begin{description}

\item{\system{SERVERTIME}}

This counter is guaranteed to be present in every implementation. The
least-significant 32-bits of the counter track the value of time used
by the server in Events and Requests.

\item{\system{FRAMEEND0}}

If present, the \system{FRAMEEND0} counter is advanced when the
display hardware for screen 0 has finished displaying a new frame.
This can be used by an application to reduce the flicker on a piece of
animation by attempting to process the graphics requests when the
display is not being updated by the hardware. This counter will only
be present if the display hardware for the screen is of the
appropriate type and if the appropriate interrupts are available to
the X server. On multi-screen displays the counters \system{FRAMEEND1}
etc. may be available.

\end{description}

Other system counters may be provided by different implementations of the
extension.  For example, a system with a good real-time clock may provide
\system{REALTIME} which contains the number of micro-seconds since midnight on
January 1, 1970. A system with flexible framestore hardware might provide a
\system{SCANLINE} counter which tracks the position of the CRT beam more
accurately than the \system{FRAMEEND} counter. This can be used to animate
more of the screen in a flicker-free manner. A video extension may wish to
define a frame counter link to an incoming analogue signal.

An ALARM is the client-side handle on an \defn{Alarm} resource.

\subsection*{Errors}

\begin{description}

\errordef{Counter}

This error is generated if the value for a COUNTER argument in a
request does not name a defined COUNTER.

\errordef{Alarm}

This error is generated if the value for an ALARM argument in a
request does not name a defined ALARM.

\end{description}

\subsection*{Requests}

\begin{description}

\requestdef{Info}

$\Rightarrow$\\
\begin{tabular}{l}
	\param{version-major},\param{version-minor}: CARD8\\
	\param{system-counters}: LISTofSYSTEMCOUNTER\\[5pt]
\end{tabular}

This request returns the version number of the extension and a list of the
system counters. No guarantees are made about the interoperability of
different versions, although versions with the same \param{version-major}
should be backwards compatible with versions with the same
\param{version-major} but a lower \param{version-minor}.

\requestdef{CreateCounter}

\begin{tabular}{l}
	\param{id}: COUNTER\\
 	\param{initial-value}: CARD64\\[5pt]
Errors: \error{IDChoice}, \error{Alloc}
\end{tabular}

This request creates a counter and assigns the identifier \param{id} to it.
The counter value is initialized to \param{initial-value} and there are no
clients waiting on the counter.

\requestdef{DestroyCounter}

\begin{tabular}{l}
	\param{counter}: COUNTER\\[5pt]
	Errors: \error{Counter}
\end{tabular}

This request destroys the given counter. All clients waiting on the
counter are released.  A counter is destroyed automatically when the
connection to the creating client is closed down if the close-down
mode is {\bf Destroy}. A \error{Match} error is generated if
\param{counter} is a system counter.

\requestdef{QueryCounter}

\begin{tabular}{l}
	\param{counter}: COUNTER\\
\end{tabular}\\
$\Rightarrow$\\
\begin{tabular}{l}
	\param{value}: CARD64\\[5pt]
	Errors: \error{Counter}
\end{tabular}

This request returns the current value of the given counter.

\requestdef{Await}

\begin{tabular}{l}
	\param{wait-list}: LISTofWAITCONDITION\\[5pt]
	Errors: \error{Counter}, \error{Alloc}, \error{Value}
\end{tabular}

This request blocks until one or more of the wait conditions in the
list is TRUE.

For each condition in the list, the current value of the counter
specified by \param{counter} is compared to \param{wait-value}. If
\param{wait-type} is \enum{Relative}, \param{wait-value} is first
incremented by the current value of the counter. The condition is TRUE
if the value of \param{counter} is greater than or equal to
\param{wait-value}.  If the value of \param{counter} is at least
\param{event-threshold} greater than \param{wait-value}, an
\event{AdvanceNotify} event is generated.

\requestdef{Advance}

\begin{tabular}{l}
	\param{counter}: COUNTER\\
	\param{amount}: CARD64\\[5pt]
	Errors: \error{Counter},\error{Value}
\end{tabular}

This request advances the given counter by \param{amount}. If the
counter reaches or passes a value for which a client is waiting, that
client is released. If the counter reaches or passes a value for which
an alarm is waiting, an \event{AlarmNotify} event is generated. A
\error{Value} error is generated if the amount is 0. A \error{Match}
error is generated if \param{counter} is a system counter. A
\error{Counter} error is generated if \param{counter} is not a valid
counter. The effect of advancing a counter to $2^{64}$ or beyond is
undefined.

\requestdef{CreateAlarm}

\begin{tabular}{l}
	\param{id}: ALARM\\
	\param{values-mask}: CARD32\\
	\param{values-list}: LISTofVALUE\\[5pt]
	Errors: \error{IDChoice},\error{Counter},\error{Value}
\end{tabular}

This request creates an alarm and assigns the identifier \param{id} to
it. The \param{values-mask} and \param{values-list} specify the
attributes which are to be explicitly initialized. The possible values
and their defaults are:

\begin{center}
\begin{tabular}{l|l|l}
Value		& Type		& Default \\
\hline
counter		& COUNTER	& \enum{None}\\
value-type	& VALUETYPE	& \enum{Absolute}\\
value		& CARD64	& 0\\
increment	& CARD64	& 1\\
limit-type	& VALUETYPE	& \enum{Absolute}\\
limit		& CARD64	& 0\\
events		& BOOL		& TRUE\\
\end{tabular}
\end{center}

If \param{value-type} is \enum{Relative}, \param{value} is incremented
by the current value of \param{counter}. If \param{limit-type} is
\enum{Relative}, \param{limit} is increment by the current value of
\param{counter}. If \param{counter} is \enum{None} and \param{value-type} or
\param{limit-type} is \enum{Relative}, a \error{Match} error is
generated.

If the \param{value} is greater than \param{limit}, or
\param{increment} is zero, or \param{counter} is \enum{None},  the
alarm state is \enum{Exhausted} and an \event{AlarmNotify} event will
be generated.  Otherwise, the alarm is \enum{Active}.

If the alarm is \enum{Active}, an \event{AlarmNotify} event will be
generated when the \param{counter} value becomes equal to or greater
than \param{value}. The \param{value} will then be incremented by
\param{increment} until it is greater than the value of
\param{counter}. If the \param{value} becomes greater than \param{limit}, the
\param{state} field of the event is set to \enum{Exhausted} and no
more events are generated.  Otherwise, \param{state} is set to
\enum{Active}.

The effect of an \param{increment} which causes \param{value} to
become $2^{64}$ or greater is undefined.

The \param{events} value enables or disables event delivery to the
client sending the request.

\requestdef{DestroyAlarm}

\begin{tabular}{l}
	\param{alarm}: ALARM\\[5pt]
 	Errors: \error{Alarm}
\end{tabular}

This request destroys an alarm. An alarm is automatically destroyed
when the creating client is closed down if the close-down mode is {\bf
Destroy}. When an alarm is destroyed, an \event{AlarmNotify} event is
generated with a \param{state} value of \enum{Destroyed}.

\requestdef{QueryAlarm}

\begin{tabular}{l}
	\param{alarm}: ALARM\\
\end{tabular}\\
$\Rightarrow$\\
\begin{tabular}{l}
	\param{counter}: COUNTER\\
	\param{value}: CARD64\\
	\param{increment}: CARD64\\
	\param{limit}: CARD64\\
	\param{events}: BOOL\\
	\param{state}: ALARMSTATE\\[5pt]
	Errors: \error{Alarm}
\end{tabular}

This request retrieves the current parameters for an Alarm.

\requestdef{ChangeAlarm}

\begin{tabular}{l}
	\param{id}: ALARM\\
	\param{values-mask}: CARD32\\
	\param{values-list}: LISTofVALUE\\[5pt]
	Errors: \error{Alarm},\error{Counter},\error{Value}
\end{tabular}

This request changes the parameters of an Alarm. All of the parameters
specified for the \request{CreateAlarm} request may be changed using
this request.

\end{description}

\subsection*{Events}

\begin{description}

\eventdef{AdvanceNotify}

\begin{tabular}{l}
	\param{counter}: COUNTER \\
	\param{wait-value}: CARD64 \\
	\param{counter-value}: CARD64 \\
	\param{time}: TIME
\end{tabular}

This event is generated when an \request{Await} request is processed.
\param{time} is the server time at which the event was generated.

\eventdef{AlarmNotify}

\begin{tabular}{l}
	\param{alarm}: ALARM \\
	\param{counter-value}: CARD64 \\
	\param{alarm-value}: CARD64 \\
	\param{state}: ALARMSTATE \\
	\param{time}: TIME
\end{tabular}

An \event{AlarmNotify} event is generated when an alarm is triggered.
\param{alarm-value} is the value of the alarm when it was triggered.
\param{counter-value} is the value of the counter when it triggered the
alarm. \param{time} is the server time at which the event was generated.

\end{description}

\end{document}
