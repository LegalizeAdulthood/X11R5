\documentstyle[twocolumn]{article}
\pagestyle{empty}
\bibliographystyle{plain}
\addtolength{\textheight}{1.7in}
\addtolength{\textwidth}{1.1in}
\addtolength{\columnsep}{0.3in}
\setlength{\oddsidemargin}{-0.1in}
\setlength{\evensidemargin}{-0.1in}
\setlength{\topmargin}{-0.5in}
\title{ XTV: A Framework for Sharing X Window Clients in  Remote Synchronous Collaboration}
\author{ 
{\it Hussein M. Abdel-Wahab}
\\
{\it Mark A. Feit}
\\[.5cm]
Department of Computer Science\\
Old Dominion University\\
Norfolk, Virginia 23529\\[.3cm]
{\it email: wahab@cs.odu.edu}
}
\date{}
\begin{document}
\pagestyle{empty}
\maketitle
\begin{abstract}
XTV is a distributed system for sharing X Window applications
synchronously among a group of remotely-located users at workstations
running X and interconnected by the Internet.  The major components of
the system are designed and implemented in such a way that make them
reusable in other collaborative systems and applications.  This paper
describes the fine technical details and knowledge required to
understand and replicate the work which went into developing XTV. 
The following concepts are discussed:
interception, distribution and translation of traffic between X
clients and display servers;
regulation of access to tools using a token passing mechanism and
reverse-translation of server traffic; and
accommodation of systems with differing architectures which may
have different byte orders for integer representation.\\[.4cm]
\end{abstract}

\noindent
{\bf KEYWORDS:} \ Distributed systems,\ Real-time collaboration,\ X
Network Protocol,\ X Window System,\ Computer-Supported Cooperative
Work, \ Computer Conferencing.

\newpage
%
% Introduction
%
\section{INTRODUCTION}

With the proliferation of high-bandwidth computer networks and the
increasing popularity and affordability of powerful workstations, it is
now feasible to provide the general user community with the environment
and tools to work together synchronously in real-time and each one can
see what other collaborators can see. This mode of collaboration is
known in the CSCW (Computer Supported Cooperative Work \cite{Greif88})
field as WYSIWIS (What You See Is What I See \cite{Stefik86}).  We
started our research in synchronous real-time group collaboration by
building the {\it Remote Shared Workspaces} (RSW) system as a research
vehicle \cite{Abdel-Wahab88}.  RSW provides the large community of  UNIX
users linked by Internet \cite{Comer90} with a general-purpose facility
that effectively converts a single-user software tool into one that can
be used for real-time collaboration by a group of remote users.  The
RSW system is restricted in that textual applications ({\it vi}
and {\it emacs}, for example) could operate with it.  This meant that
output could contain no graphics and input could only be taken from the
keyboard.  RSW's major advantage, however, is that a user equipped only
with a dumb terminal can participate and collaborate with others.

Our current system, called {\it X TerminalView} (XTV), is an extension
of RSW using MIT's X window system \cite{Scheifler86} where individual
X clients (also called {\it applications} or {\it tools}) can be
shared.  X is gaining ground as a standard windowing system and is
currently supported by most leading workstation manufacturers including
Sun, DEC, IBM, HP, Apple, AT\&T.  Because it is designed as
a network-transparent, device-independent windowing and graphics
system, X completely achieves its device independence by splitting the
job of drawing windows into two parts using the increasingly familiar
client/server model.  One or more clients, connected through a network,
communicate with the server by sending packets of instructions and data
(called {\it requests}) conforming to the X protocol \cite{Nye890}.  X
can be used for truly distributed systems since a client running on one
workstation can open windows on remote workstations which may have been
made by different manufacturers.  It is these aspects of X that make it
an appealing and valuable system for supporting group collaboration. 

XTV has been designed in such a way that most of its components can be
reused in building other collaborative tools.  For example, the core
routines which handle the translation of X protocol requests and
replies can be easily extracted and implanted in other programs which
might require this capability.  XTV and all of the methods developed
during its implementation are in the public domain, and all attempts
have been made to reveal as many of the its operational details so that
others may replicate our work or use it in other contexts.

Section 2 explains how XTV collaborative sessions are started and how X
clients are invoked during the session.  This section also describes
the user interface to the system and the relationship between the
distributed processes created during a session and the communication
paths between them.  Section 3 describes the process of intercepting
traffic between clients and servers and the methods used to examine
this traffic and route it to the proper destinations.  Also outlined
are the procedures required for overcoming the differences in the
operational characteristics used by different X servers.  Section 4
details the translation required to make request packets intended for
one server indicate the correct actions to another.  The reverse
process, translating events and replies from multiple servers to provide the right
information to a single client, is shown next.  Also described here are
the methods used to overcome the problem of communicating with
processes on different hosts that may use different byte orders to
represent integer quantities.  Finally, Section 5 outlines the
conclusions drawn from the development of XTV and the plans for its
future extension.

%
% System Description
%
\section{SYSTEM DESCRIPTION}
\subsection{Starting a New Session}

Figure 1 shows the overall structure, relationship and communication
paths between different processes involved in a given collaborative
session.  
To start a session, one
user, designated the ``chairman,'' creates a ``Session Manager''
process.  This process is invoked on the local host or any remote host
able to connect to the local host's display server.  The Session Manger
listens for connections from ``Participant Agent'' processes at a
TCP/IP port which is well-known to all other copies of XTV.  A
participant joins an ongoing session by creating a Participant Agent
process and specifying as arguments the host and session ID number
(referred to hereafter as {\tt SessionNumber}) of the Session Manager.

Each participant is presented with a ``Session Panel'' similar to that
shown in Figure 2.  The title section contains information about the
participant (e.g, the name is {\tt wahab}, connected to 
session number {\tt 2} at {\tt 9:15am} from host {\tt seth.cs.odu.edu}).  On the
session panel are three buttons:

\begin{itemize}

\item {\it New Tool} - Brings up a panel used to start a new X client
(tool) to be shared among participants in the session.
\item {\it Leave Session} - Leaves the session without any disruption to
other participants.
\item {\it End Session} - Brings the session to an end by terminating
all tool and agent processes, destroying windows at all hosts
associated with it and closing all network connections.
\end{itemize}

The {\it Participants} area of the panel contains a list of all of the
session's participants.  The user can open a window containing
pertinent information on a particular participant by clicking on his or
her name.  By clicking on the {\it Talk} button, a private, one-to-one
conversation can be carried on with that participant.

The {\it Tools} area contains the names of all tools being used in the
session.  Clicking on the name of a tool opens a ``tool panel'' window
with some of the tool's operating information and mechanisms for
manipulating its associated ``token.''  By using a first-come,
first-served method for handling the token, regulation of which
participant may provide input to each tool is achieved.  Several
buttons allow participants to control a tool and its token:

\begin{itemize}

\item {\it Get Token} - Places the participant at the end of the queue
to get the token.  If no participant possesses the token, it is passed
immediately.

\item {\it Urgently Get Token} - Bypasses the token queue and passes
the tool's token directly to the participant.  At present, this
function is restricted only to the session chairman.


\item {\it Drop Token} - Passes the token on to the next participant in
line for the token.  If the token queue is empty at that point, none of
the participants may provide input until someone clicks the {\it Get
Token} button.

\item {\it Leave Tool} - Disconnects the participant from the tool.
When this happens, the participant's windows associated with
the tool  are closed while others continue to use the tool.

\item {\it Join Tool} - Allows a participant to join a tool already in
progress.  This function is not currently supported, as a great deal of
state information maintained by the server would need to be kept by XTV
as well.  The ability to acquire the information stored by the server
will become practical only when a future version of the X protocol
supports it.

\item {\it End Tool} - Terminates the tool and all associated
processes, windows and connections to all participating hosts.

\end{itemize}

The {\it Token Queue} area of the tool panel indicates which users are
participating in this tool, who possesses the token and who is queued
for it.
Note that it is not necessary for all session participants to join a
given tool, some of them may elect not to participate in some of the
invoked tools.


%
% Starting a New Tool
%
\subsection{Starting a New Tool}

To start a new tool, one of the participants clicks on the {\it New
Tool} button of the session panel.  Through a pop-up dialog box,
the name of the tool, its command line arguments and
place of execution are specified.  XTV executes tools
at the chairman's location or on systems accessible through the Internet.
Starting a new tool creates the following processes (See Figure 1):

\begin{itemize}

\item {\it Packet Switcher} - Responsible for handling the initial
connection from the X client to the chairman's X server and to all of
the participants' packet translators.  Once those connections are
established, the packet switcher handles the distribution of all
traffic from the client to the server and the other packet translators
as well as multiplexing input from all servers back to the client.
This process interacts with the token manager (below) to determine the
disposition of incoming traffic from the servers.

\item {\it Packet Translator} - Converts client-to-server and
server-to-client traffic from one server's format to that of another.
Any packet received from the Packet Switcher is
``forward'' translated  and sent to its
local server. If the associated participant has the token, then any
event or reply received from its local X server will be ``reverse''
translated and send to the Packet Switcher.
(The translation processes are described in Section 4).

\item {\it Token Manager} - Accepts input from all token agent
processes (below) and determines the disposition of the token, which
can be accomplished using any desired token management
protocol \cite{Abdel-Wahab88} and informs the packet switchers of its
new state.

\item {\it Token Agent} - Maintains the token panel and implements all
functions associated with the panel buttons by interacting with the
token manager.

\end{itemize}
%
% Connection Setup
%
\section{CONNECTION SETUP}

\subsection{Intercepting traffic between X clients and servers}

IN X, the specification of the display server that a  client is to use 
takes the
format {\tt {\it hostname}:{\it display}.{\it screen}}.  The {\it hostname}
is a standard Internet host name such as {\tt seth.cs.odu.edu} or {\tt
128.82.7.4}.  Generally, {\it display} and {\it screen} are both zero,
indicating that logical screen number zero of physical display number
zero is to be used.  This information is either placed in the {\tt
DISPLAY} environment variable or specified using arguments supplied on
the command line.

The X display server accepts connections from clients by listening to a
TCP/IP socket at a port well known to all clients (defined by the
protocol as {\tt X\_TCP\_PORT}).  
Clients contact the server 
waiting at {\tt X\_TCP\_PORT + DisplayNumber}.  
Thus if {\tt DisplayNumber} is 0, the client contacts the 
X display server.
XTV's approach to
intercepting the traffic between the client and server is based on
providing false display information to the client (by way of {\tt
DISPLAY} environment variable).  A TCP/IP socket at the port the client
will attempt to connect with (using its ``false'' information) is
opened by the XTV packet switcher, and thus the client connects to a packet 
switcher instead of the
``real'' X server.  For example, if XTV is running on host {\tt seth} and its
{\tt SessionNumber} is {\tt 2}, XTV will open its port at {\tt
X\_TCP\_PORT + 2} and set {\tt DISPLAY} to {\tt seth:2.0}.
Using this example, XTV would, in effect, perform these {\tt csh}
commands to invoke {\it xcalc}:
\begin{quote}
{\tt \% setenv DISPLAY seth.cs.odu.edu:2.0} \\
{\tt \% xcalc}
\end{quote}

%
% Connection Request & Reply
%
\subsection{Connection Request and Reply}

Once a connection is established with a client, the packet switcher
waits for a ``connection request'' packet, the first data sent by
clients conforming to the X protocol.  This packet contains the
following information:

\begin{itemize}
\item {\it ByteOrder} - A one-byte value indicating how integer
quantities are to be interpreted by the display server.  This is to
accommodate architectures which have varying methods of storage.  The
two alternatives are LSB first (called {\it big endian} and used by
IBM, Sun workstations) and MSB first (called {\it little endian} and
used by DEC workstations).
\item{\it Protocol Version} - Indicates to the server which version of
the X protocol it is expecting the server to use (e.g., Version 11,
Release 4).
\item{\it Authorization Data} - Designed for use in authentication of
server connections but considered meaningless in the present version of
the protocol.
\end{itemize}

\noindent

The chairman's packet switcher accepts this information and
forwards a copy it to its X server (called the {\it primary} server)
and to the packet translators of all participants.  Each packet translator
in turn forwards this packet to their own X servers (called {\it local}
servers).

If an X server accepts a connection request, it replies by sending back
a ``connection acceptance'' packet, from which the chairman's packet
switcher extracts and passes on the following information to the
participants:

\begin{itemize}
\item {\it Resource ID Base}
\item {\it Resource ID Mask}
\item {\it Root Window ID}
\item {\it Default Colormap}
\item {\it Root Visual}
\item {\it Visual ID List}
\end{itemize}

\noindent
This information is used in translating requests and events from the
local server's format to that of others and back.  (The details of this
procedure are described in section 4)

When a participant's packet translator receives a connection acceptance
packet from its local server, the same information is extracted and
stored for use in translation.  Since there is no ``real'' client to
forward this information to, the packet is then discarded.

%
% Packet Translation
%
\section{PACKET TRANSLATION}

\subsection{Translating Resource IDs}

X identifies server-stored abstractions (such as windows, colormaps and
atoms) using a 32-bit identifier called a {\it resource ID}.  Resource
IDs are assigned by the client using a base value and bitmask provided
by the server during the connection process.  The bitmask indicates
which bits may be used by the client in creating a resource ID which is
unique.  Once those bits have been filled in, the base value is
overlaid and the completed ID is passed to the server.

By making use of the values and masks of the primary server and a local
server, it is possible to translate resource IDs from one to the other
without sacrificing its uniqueness.  This is accomplished by removing
the base value from the source ID, by ANDing it with the server's bitmask
and ORing the result with the target ID and overlaying the target
server's base value.

For example, assume the source server has provided a client with a
resource ID base of {\tt 0xabc00000} and a bitmask of {\tt 0x000fffff}
and our target server has provided  {\tt 0xdef00000} 
for the base value. Using the above steps,
the resource ID {\tt 0xabc12345} would translate to {\tt 0xdef12345}.

It is worth mentioning that not all resource IDs are translated in this
manner.  Certain items such as the root window, default colormap and
visual IDs are assigned by the server and do not necessarily conform to
a particular base value and bitmask.  To accommodate these cases, the
necessary values are extracted from the connection acceptance packet
and stored for use by the translation routine.  In the interest of
speed, incoming resource IDs are first checked to see if there is a
match between it and the source server's base value.  If there is, the
procedure described above is used.  Otherwise, the window, default
colormap and visual IDs are checked manually and the corresponding
value for the target server is returned.


\subsection{Translating Client Requests}

All requests submitted to the X server by clients begin with a header
whose format is specified by the X protocol.  In this header is the
following information:

\begin{itemize}

\item {\it Major Opcode} - An unsigned short integer (one byte)
indicating which server function is being requested.  Values {\tt 0}
to {\tt 127} represent the standard functions built into all servers.
Those numbered {\tt 128} through {\tt 255} are reserved for
implementation-specific extensions.


\item {\it Auxiliary Field} - An unsigned short integer (one byte) used
as a data field in requests which pass only one byte of data aside from
the major opcode and request length.  This eliminates the need for
extra network traffic to complete short requests.

\item {\it Request Length} - An unsigned integer (two bytes)
indicating how many four-byte units of data the request contains,
including the header, which is four bytes long.
\end{itemize}

Once a complete request arrives at the chairman's packet switcher,
copies are distributed unmodified to the primary server and the packet
translators of all participants.  At the participant's end, the contents
of the request header are examined and one of the following actions is
taken to translate the packet from the primary server's format to that
of the local server:


\begin{itemize}

\item {\it No Translation} - Requests such as {\tt Bell} (shown in
Figure 3a) which contain no resource IDs require no translation and
are passed unmodified to the local server.

\item {\it Simple replacement} - Requests such as {\tt DestroyWindow}
(shown in Figure 3b) contain resource IDs in fixed locations which can
be translated in a straightforward manner using the methods previously
described.

\item {\it Complex replacement} - Requests such as {\tt CreateWindow}
(shown in Figure 3c) have a variable length which is determined by the
contents of certain elements in a part of the packet which is known to
be fixed.  In these cases, a special function capable of handling that
particular request is called.
For exampl, the {\tt VALUE LIST} of the {\tt CreateWindow} packet
shown in Figure 3c requires special processing (the number of 
elements of the list is equal to the number of 1s in the 
{\tt value mask} field).

\end{itemize}

For the sake of efficiency, the entire request translation operation is
table-driven.  When a packet arrives, its opcode is used to directly
access a table containing information on what action should be taken.
Each record in the table contains the following fields:

\begin{itemize}

\item {\it Packet Name} - An ASCII representation of the standard name
of the request as assigned by MIT.  The contents of this field are
presently used for debugging purposes only.

\item {\it Resource ID Offsets} - A zero-terminated list of the offsets
(in units of four bytes) inside the request containing resource IDs,
which are always at a four-byte boundary.  Negative numbers indicate
that a resource ID found is to be left alone if the value of the
resource ID is 0 or 1, translated otherwise. If there are no
replacements to be done, the first item in the list is a zero.

\item {\it Special Function} - A pointer to a function designed for
handling translation of a particular type of request.  If no special
translation is required, this pointer is {\tt NULL}.  In the case of
requests which are to be eliminated entirely (such as those which would
generate useless replies at a participant's local server), a function
called {\tt ForceDropDead} is called.  {\tt ForceDropDead} replaces the
request's major opcode with {\tt 127} (the {\tt NoOperation} request),
and the packet is not transmitted to the participants.

\end{itemize}

\noindent
For example, the entry numbered {\tt 001} in the table contains the
following:

\begin{quote}
{\tt \{ "CreateWindow", \{1,2,-6,0\}, XlateCreateWindow \}}
\end{quote}

\noindent

When a {\tt CreateWindow} request arrives from the
Packet Switcher, its record is retrieved from the table.  The Resource
IDs at offsets of 4, 8 and 24 bytes will be translated to match the
local server, with the Resource ID at bytes 24-27 remaining
untranslated if they are 0 or 1.  Once the fixed translations are
complete, the request is passed to a function called {\tt
XlateCreateWindow} which handles all translations on the variably-sized
portion of the request.

\subsection{Translating Server-Generated Events and Replies}

Much like requests passed from the clients to the X server, events
and replies generated by the local servers of the participants must be translated
from the local format to that used by the primary server. The same
methods used in translating requests are applied in reverse, the only
significant difference being that a different table is used to
determine the course of action.

Processing server-generated events and replies requires special consideration on
the part of the participants' packet translators.  Events must be
carefully filtered to avoid confusing the X client program.  Generally,
the one variable governing how events are filtered by a particular
participant is whether or nor it has the token.  When the token is not
available, these types of events are discarded:

\begin{itemize}
\item Keyboard activity
\item Pointer activity
\item Configuration notification
\end{itemize}

To maintain complete, accurate displays of windows for all of the
participants' screens, exposures are always propagated.  

To avoid irrational behavior on the part of XTV clients, participants
never pass any of these events back to the client:

\begin{itemize}
\item Reparenting notification
\item Keymap notification
\end{itemize}

As each event arrives, any fields containing resource IDs must be
translated from the format used by the participant's local server to
that of the primary server.  This is easily accomplished, as the
participant already has copies of both formats on hand and can pass
them to the resource ID translator in ``reverse'' order (i.e., passing
the local server's information to the resource as the ``source''
argument and the primary server's information as a ``target''
argument).  This effectively ``back-translates'' each of the resource
IDs contained in the event.  On a larger scale, the translation of
event packets is handled in much the same way as requests.  A separate
translation tables for events and replies is maintained and like the first, it
provides the same information about the locations of resource IDs
within the packet and pointers to special functions required for
processing particular events and replies.

\subsection{Byte Order Interpretation}

Under the rules of the X protocol, client processes may run on machines
with differing byte orders for the representation of integers.

For Example, we may run an X client on a DECStation (MSB first), the
chairman's packet switcher on an IBM RISC/6000 (LSB first) and a
participant's packet translator on a Sun SPARCStaion (also LSB first).
In such a heterogeneous environment, each process must correctly
interpret the integer quantities it receives from the others.

At startup, the packet switcher finds the byte order used by the host
on which it is running.  This is accomplished by taking known integer
value and examining the way it is stored as bytes.

XTV takes a hands-off approach to handling the byte order of each tool.
Whenever the need arises to interpret one of the client's integers, the
value and the client's byte order are first passed to one of {\tt
SwapInt()} or {\tt SwapLong()} for processing before the value is
placed elsewhere and then actually used.  If there is a difference
between the local and client byte orders, the contents of the integer
are rearranged to properly express the integer locally.  

In addition to dealing with each tool's requests and events by using
its byte order as a guide, XTV also makes use of that information to
facilitate communication between the chairman's packet switcher and
the participants' translators.  Since situations requiring the transfer of
integers only arise when dealing with tools, all communications are
done in the byte order of the particular tool in question.  By doing
this, the need for the chairman's packet switcher to be aware of the
byte orders of the participants and vice-versa is eliminated.

%
% Future Goals
%
\section{CONCLUSIONS AND \ \ \ \ \ \ \ \ 
FUTURE GOALS}


This paper has presented a framework to allow a group of users to join
in a collaborative session for viewing and interacting with X-based,
single-user applications.  In effect, XTV converts the tools we are all
familiar and comfortable with from single- to multi-user without
modifications to source code, libraries, application behavior, the
operating system, or X servers.

The token and participant management policies are designed as
independent components of the system which can be modified to suit the
requirements of the group.  The packet translation tables and routines
were designed for efficiency and need not be changed as long as there
is no change in the X protocol. However, they can be easily extended to
include any extensions to the protocol, such as the addition of new
request types.

XTV is by no means complete.  In its present state, it is not much more
than a working prototype which provides the features described here.
The current system is vulnerable to a failure of the chairman's packet
switcher or token manager, either of which will bring the tool to a
halt. 
One possible way of dealing with this problem is
the implementation of a replicated tool architecture similar to the
ones described in \cite{Abdel-Wahab88,Lantz86}.

XTV's performance in its current implementation
has been more than adequate on medium-speed local area networks.  A
test of the system over a FDDI network is planned for the future, and
it appears that these higher-bandwidth channels will result in a better
performance and help to eliminate the occasional time lag experienced.

A major function remaining to be implemented is to allow users
to join a tool after it has already started.  As discussed earlier,
future extensions to the X protocol should make this facility not only
feasible but practical as well.\\[1cm]

\noindent
{\large \bf ACKNOWLEDGEMENTS}\\

We want to extend special thanks to Don Smith, Peter Calingaert
and Kevin Jeffay of UNC for their invaluable assistance and encouragement
at various stages of this project. 
We are greatful to Alan Blatecky and Tom Sandoski of MCNC for testing and 
experimenting with different versions of XTV.
We also would like to thank Ashit Patel, John Menges, Alok Ramsisaria,
Jin-Kun Lin and Jie-Shan Lin of UNC, and
Osman ZeinELDine, Mohamed ElToweissy  and Rafat Shaheen of ODU
for their contributions and comments on various aspects of XTV.
\newpage
%
% Bibliography
%
\begin{thebibliography}{}

\bibitem[Abdel-Wahab 88]{Abdel-Wahab88}
Abdel-Wahab, H. M., Guan, Sheng-Uei and Nievergelt, J. [1988] Shared
Workspaces for Group Collaboration: An Experiment using Internet and
UNIX Interprocess Communications, {\it IEEE Communications Magazine},
Vol. 26, No. 11, 10-16 (Nov. 1988).

\bibitem[Comer 90]{Comer90}
Comer, D., [1990] {\it Internetworking with TCP/IP: Principles, Protocols and
Architecture}, Second Edition, Prentice-Hall, 1990

\bibitem[Greif 88]{Greif88}
Greif, I., [1988] {\it Computer-Supported Cooperative Work: A Book of Readings},
Morgan Kaufmann Pub. Co., Palo Alto, CA (May 1988).

\bibitem[Lantz 86]{Lantz86}
Lantz, K. A., [1986] An Experiment in Integrated Multimedia Conferencing,
{\it Proceedings, Conference on Computer-Supported Cooperative Work,}
Austin, Texas, 267-275 (Dec. 1986).

\bibitem[Nye 89]{Nye890}
Nye, Adrian, [1989] {\it X Protocol Reference Manual for Version 11},  Volume 0,
O'Reilly \& associates, Inc., Sebastopol, CA (1989).

\bibitem[Scheifler 86]{Scheifler86}
Scheifler, R. W. and Gettys, J. , [1986] The X window system, {\it ACM
Transactions on Computer Graphics} 79-109 (May 1986).

\bibitem[Stefik 86]{Stefik86}
Stefik, M., Bobrow, D. G., Lanning, S., Tatar, D., Foster, G. [1986] WYSIWIS
revised: Early Experience with Multi-user Interfaces, {\it
Proceedings, Conference on Computer-Supported Cooperative Work,}
Austin, Texas, 276-290 (Dec. 1986).

\end{thebibliography}
\end{document}
